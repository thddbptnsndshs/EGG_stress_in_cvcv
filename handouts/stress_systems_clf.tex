% !TEX TS-program = XeLaTeX

% STYLE

\documentclass[a4paper, 12pt]{article}
\usepackage[left=3cm,
		    right=3cm,
    		    top=3cm,
		    bottom=3cm,
		    bindingoffset=0cm]{geometry}
		    \usepackage{array}
\usepackage{float}
\usepackage{graphicx}
\usepackage{hyperref}
\hypersetup{
    colorlinks=true,
    linkcolor=black,
    citecolor=black,
    filecolor=black,
    urlcolor=black,
}
\graphicspath{ {./images/} }
\usepackage{subfig}
%\usepackage{enumerate}
\usepackage[normalem]{ulem} % underlining
\usepackage{booktabs} % tables
\PassOptionsToPackage{table}{xcolor}% coloring tables
\usepackage[shortlabels]{enumitem}
\setlist[enumerate]{itemsep=-3pt}

% LANGUAGE + FONT
		    
\usepackage[english]{babel}
\usepackage[backend=biber,
                     style=unified]{biblatex}
\newcommand{\citeay}[2][]{
   \citeauthor{#2} (\citeyear[#1]{#2})}
\addbibresource{ref.bib}
\usepackage{fontspec}  
\usepackage{pifont}
\setmainfont{Minion 3}

% DRAWING

\usepackage{tikz}
\usepackage{tikz-qtree}
\usetikzlibrary{shapes.geometric}
\usetikzlibrary{trees,arrows}
\usetikzlibrary{positioning}

% LINGUISTICS 

%\usepackage{gb4e}
\usepackage{expex}
\usepackage[glossaries]{leipzig}
\makeglossaries
%\newleipzig {indef} {indef} {Indefinite}

\title{What stress generally is}
\author{Sasha Shikunova}
\date{EGG2023, last updated \today}

\begin{document}

\maketitle

%- What is prosody about: tone, pitch accent and stress (we focus on stress)
%- Key properties of stress -- obligatoriness and culminativity, contrast between stress and tone as prosodic properties of strings
%- Examples of tonal/pitch accent/free-stress/stress languages (the continuum)
%- What stress can count -- syllables, moras and feet
%- Two approaches to stress placement: metrical theory and primary-accent-first theory (foot structure vs assigning primary accent first)
%- Possible reading: van der Hulst, H. (2009). Brackets and grid marks, or theories of primary accent and rhythm. Contemporary views on architecture and representations in phonological theory, 225-245.

		\section{Stress is part of prosody, right?}
		
%	(scale drawing)

	Following \textcite{hyman2006}: 
	
	\begin{enumerate}[$\gg$]
		\item One end of the prosody scale is tone and the other is stress accent
		\item There are systems in between that are not best categorised as either prototype
		\item In-between is sometimes called \emph{pitch accent}
	\end{enumerate}
	
	\pex\label{def:stress}A language with stress accent is one in which there is an indication of word-level metrical structure meeting the following two central criteria:
		\a \textsc{obligatoriness}: every lexical word has at least one syllable marked for the highest degree of metrical prominence (primary stress);
		\a \textsc{culminativity}: every lexical word has at most one syllable marked for the highest degree of metrical prominence. \trailingcitation{\parencite[p. 231]{hyman2006}}
	\xe
	Why is the `exactly one stress per word' property split into \textsc{obligatoriness} and \textsc{culminativity} in Hyman's definition?
	
	\ex
		A language with tone is one in which an indication of pitch enters into the lexical realisation of at least some morphemes. \trailingcitation{\parencite[p. 1367]{hyman2001}}
	\xe
	
	\begin{enumerate}[$\gg$]
		\item Both tone and stress are properties of syllables. Why not of vowels?
		\item Tone is featural; there may be several different tones (e.g. H/L)
		\item Stress is prominence; can there be several different stresses?
		\item Which one of the properties in (\ref{def:stress}) is therefore more important?
	\end{enumerate}
	
	Apart from pure tone and pure stress, there is pitch accent (PA), which supposedly is in the middle. Some syllables are lexically marked with \emph{accents}, which influence pitch placement. Similarly, accents can mark prominence. How do we tell PA from stress accent (SA)?
	
	\begin{enumerate}[$\gg$]
		\item If a language uses PA, the use of pitch elsewhere should be restricted
		\item If a languages uses SA, pitch can be used elsewhere, say, for phrase-level intonation
		\item Also, SA is phonetically different, i.e. not exponed by pitch alone (like in PA systems)
	\end{enumerate}
	
	We will be primarily interested in stress.
	
		\section{What stress can count}

	Syllables:
		
	\begin{enumerate}[$\gg$]
		\item Each one has a nucleus -- usually something sonorant, like a vowel
		\item Optionally, there is an onset or a coda
		\item Varying syllable structures; languages can tolerate very few to almost any structure
		\item Depending on the structure, syllables can be \emph{heavy} or \emph{light}
	\end{enumerate}
	
	\pex
		Example rule: stress the penultimate syllable (Polish)
			\a Ag$\cdot$\textbf{niesh}$\cdot$ka
			\a u$\cdot$ni$\cdot$wer$\cdot$si$\cdot$\textbf{te}$\cdot$tu
	\xe
	
	Moras (as in Metrical stress theory, \cite{hayes1995}):
	
	\begin{enumerate}[$\gg$]
		\item Less than syllables; syllables consist of 1 or 2 moras
		\item Heavy syllables have 2 moras, light syllables have 1
		\item Another version of the rule for Latin could be: stress the syllable containing the antepenultimate mora
		\item Heavy syllables (closed or containing long vowels)
	\end{enumerate}
	
\begin{table}[H]
\centering
\begin{tabular}{ll}
\textbf{pa}$\cdot$pa        & Eu$\cdot$\textbf{rō}$\cdot$pa         \\
fi$\cdot$\textbf{gū}$\cdot$ra     & per$\cdot$\textbf{sō}$\cdot$na        \\
pan$\cdot$\textbf{thē}$\cdot$ra   & hal$\cdot$lū$\cdot$ci$\cdot$\textbf{nā}$\cdot$tus \\
me$\cdot$\textbf{mo}$\cdot$ri$\cdot$a   & \textbf{si}$\cdot$mi$\cdot$lis        \\
pa$\cdot$ra$\cdot$\textbf{dox}$\cdot$us & fun$\cdot$dā$\cdot$\textbf{men}$\cdot$tum   \\
ū$\cdot$ni$\cdot$\textbf{cor}$\cdot$nis & \textbf{cal}$\cdot$ci$\cdot$trō      
\end{tabular}
\caption*{Latin: stress the penultimate syllable; if short, stress the antepenultimate syllable [\href{https://en.wikiversity.org/wiki/Latin/Pronunciation\_Stress}{source}]}
\end{table}
	

	Feet:
	
	\begin{enumerate}[$\gg$]
		\item Composed of 1 or 2 syllables
		\item One metrically prominent syllable per foot
		\item A syllable that is not in a foot cannot be stressed (is \emph{extrametrical})
	\end{enumerate}
	
			\subsection{How accent placement is determined}
			
	There can be parametric variation wrt. foot parsing and stress assignment \parencite{hayes1980}
	
	\begin{enumerate}[$\gg$]
		\item Split into feet: \emph{right-to-left/left-to-right}
		\item Find the head syllable: \emph{left-headed/right-headed}
		\item Find the head foot: \emph{leftmost/rightmost}
		\item For penultimate stress there is extrametricality -- one syllable is not counted
	\end{enumerate}
		
\begin{table}[]
\centering
\begin{tabular}{ll}
ˌpăsaˈnɛma &  (pă$\cdot$sa)$\cdot$(\textbf{nɛ}$\cdot$ma)\\
ˈλaraś     &  (\textbf{λa}$\cdot$raś)\\
ˈλaraśa    &  (\textbf{λa}$\cdot$ra)$\cdot$<śa>
\end{tabular}
\caption*{Khanty: right-headed word structure, right-to-left left-headed feet}
\end{table}
				
		\section{Primary accent vs foot structure: what comes first}
		
	Every word (in some languages, at least) has a primary accent. Does it make a difference, whether the primary accent or the foot structure is determined first in the computation of stress? 
	
	\textcite{vanderhulst2009} argues against Metrical theory, which says that feet are computed before the primary stress.
	
	\begin{enumerate}[$\gg$]
		\item Rhythmic accent is secondary to the primary accent
		\item This is controversial: what if a system exists where the order can only be the reverse?
	\end{enumerate}
	
	There are, however, several reasons to believe in the primary-accent-first theory:
	
	\begin{enumerate}[i.]
		\item Different criteria for assigning primary and rhythmic stress
		\item Foot structure for primary stress from one side, rhythmic from the other -- two distinct foot types?
		\item Different foot headedness
		\item Primary accent can be lexically set, not so with rhythmic stress \hfill \parencite{vanderhulst2009}
	\end{enumerate}
	
				\subsection{What can a primary-accent-first system generate?}
				
	The primary stress assignment process works via weight projection to a suprasegmental level:
	
	\begin{enumerate}[a.]
		\item Project syllable weight to level 1: \emph{yes/no} 
		\item If level one is empty -- default projection strategy: \emph{leftmost/rightmost} syllable 
		\item Primary accent assigned to \emph{leftmost/rightmost} syllable projected to level 1 
	\end{enumerate}
	Heavy syllables project in weight-sensitive languages; also, projecting syllables can be diacritically marked. We already have $2 \cdot 2 \cdot 2 = 8$ options, but that is not all parametrisation available. Consider the domain settings:
	
	\begin{enumerate}[a.]
		\item \emph{Bounded/unbounded domain} -- the whole word versus two edge syllables (a foot)
		\item If bounded, \emph{leftmost/rightmost} foot
		\item Extrametricality -- one syllable does not matter for stress assignment: \emph{yes/no}
		\item If extrametricality: extrametrical syllable to the \emph{left/right} of the domain
	\end{enumerate}
	This gives us $8 \cdot 3 \cdot 3 = 72$ possible systems. Weight-sensitive systems are rather restricted, whereas if special syllables can be diacritically marked, the lexical arbitrariness makes many more systems possible.
	
				\subsection{What can't a primary-accent-first system generate?}
				
	A so-called \emph{count system}, where primary accent crucially depends on the word being fully parsed into feet: say, penultimate stress in words with $2N$ syllables and final stress in words with $2N + 1$ syllables.
	
	\pex A made-up count system
		\a ka$\cdot$ga$\cdot$\textbf{gu}$\cdot$ga
		\a u$\cdot$ga$\cdot$\textbf{ga}
	\xe
	The workaround: count systems may actually lack primary accent, so stress assignment only proceeds via feet parsing without placing any primary accent. How such situations may come to be, according to \textcite{vanderhulst1997}:
	
	\begin{enumerate}[$\gg$]
		\item Historical change from one anchor side to the other
		\item Fragile notion of word in polysynthetic languages: stress more like phrasal accent
	\end{enumerate}
	Why can't we have both `metrical' languages and primary-accent-first languages?
	
		\section{Summary}
		
	\begin{enumerate}[$\gg$]
		\item Stress contrasts with tone and pitch accent
		\item Primary accent can be different from rhythmic accent in important respect, that's why it makes sense to assign primary accent first
		\item Some systems may have no primary accent whatsoever
	\end{enumerate}
	
			\subsection{Next up...}
			
	We have located stress among prosodic phenomena and reviewed some stress assignment algorithms, but there is much more to it:
	
	\begin{enumerate}[$\gg$]
		\item Phonetic and phonological exponents of stress
		\item How exactly stress and syllable weight can interact
		\item How stress assignment domains are formed
	\end{enumerate}
		
\printbibliography
\end{document}